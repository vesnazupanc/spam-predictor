% when converting Markdown files, use the -H option to refer the head.tex file

\usepackage{fancyvrb,newverbs,xcolor}
\usepackage{booktabs}

\usepackage{hyperref}
\usepackage[utf8]{inputenc}

% \hypersetup{
%  pdftitle={Navodila za pripravo dokumentacije - sphinx},
%  pdfauthor={Vesna Zupanc},
%  pdfsubject={Dokumentacija},
%  pdfkeywords={dokumentacija,sphinx,projekt}
%}

\usepackage{fancyhdr}
\pagestyle{fancy}


\usepackage{graphicx}
\usepackage{float}
\floatplacement{figure}{H}


% change background color for inline code in
% markdown files. The following code does not work well for
% long text as the text will exceed the page boundary
\definecolor{bgcolor}{HTML}{F4F4F4}
\let\oldtexttt\texttt

\renewcommand{\texttt}[1]{
  \colorbox{bgcolor}{\oldtexttt{#1}}
  }


% Change the default style of block quote
% \usepackage{framed}
% \usepackage{quoting}

% \definecolor{bgcolor}{HTML}{DADADA}
% \colorlet{shadecolor}{bgcolor}
% define a new environment shadedquotation. You can change leftmargin and
% rightmargin as you wish.
% \newenvironment{shadedquotation}
%  {\begin{shaded*}
%   \quoting[leftmargin=1em, rightmargin=0pt, vskip=0pt, font=itshape]
%  }
%  {\endquoting
%  \end{shaded*}
%  }

%
% \def\quote{\shadedquotation}
% \def\endquote{\endshadedquotation}


% Start a new page after toc
\let\oldtoc\tableofcontents
\renewcommand{\tableofcontents}{\oldtoc\newpage}